\documentclass[russian, english]{article}

\usepackage[T2A]{fontenc}
\usepackage[utf8]{inputenc}
\usepackage[english,russian]{babel}
\usepackage{amsmath}
\usepackage{pgfplots}
\usepackage{graphicx}
\usepackage{csvsimple}
\usepackage{hyperref}

\inputencoding{utf8}
\def\code#1{\texttt{#1}}

\begin{document}

\begin{titlepage}
\centering
	{\scshape\LARGE Методы оптимизации \par}
	\vspace{1cm}
	{\scshape\Large Лабораторная работа №3\par}
	\vspace{2cm}
	{\Large\itshape Дмитрий Проценко M3234 \par
	Кирилл Прокопенко M3236 \par
	Николай Холявин M3238 \par}
	\vfill
	ИТМО y2019
	\vfill
	{\large \today\par}
\end{titlepage}

\tableofcontents
\newpage

\section{Цели}
\begin{itemize}
	\item реализовать $LU$ разложение и решение систем линейных уравнений на его основе
	\item оценить зависимость погрешности от числа обусловленности, размерности (на сгенерированных с диагональным преобладанием, матрицах Гильберта)
	\item сравнить с другими методами (Гаусса)
	\item реализовать профильный формат хранения матрицы
	\item (бонус)!!!
\end{itemize}
\section{Теория}
\subsection{Профильный формат матрицы}
Отдельно хранится главная диагональ и 2 треугольника, при этом треугольники хранятся в следующем формате: нжний левый хранится по строкам, верхний правый по столбцам, индексы первых элементов соответствующийх линий совпадают. Для улучшения кэширования и уменьшения фрагментации памяти каждый треугольник имеет общую память.\\
Это может выглядеть, например, так:

\newcommand*{\GridSize}{8}

\newcommand*{\ColorCells}[1]{% #1 = list of x/y/color
	\foreach \x/\y/\color in {#1} {
		\node [fill=\color, draw=none, thick, minimum size=1cm] 
			at (\x-.5,\GridSize+0.5-\y) {};
	}
}

\newcommand*{\ColorCellsdxdy}[4]{% #4 = list of x/y/color
	\foreach \myi in {1,2,...,#1}{
		\foreach \x/\y/\color in {#4} {
			\node [fill=\color, draw=none, thick, minimum size=1cm] 
				at (\myi*#2-#2+\x-.5,\myi*#3-#3+\GridSize+0.5-\y) {};
		}
	}
}
\begin{center}
\begin{tikzpicture}[scale=1]
	\begin{scope}[thick,local bounding box=name]
		% \ColorCells{1/1/blue, 2/3/red, 3/2/green, 4/4/yellow}
		\ColorCellsdxdy{\GridSize}{1}{-1}{1/1/red}
		\ColorCellsdxdy{3}{1}{0}{2/5/blue}
		\ColorCellsdxdy{3}{0}{-1}{5/2/blue}
		\ColorCellsdxdy{5}{1}{0}{1/6/green}
		\ColorCellsdxdy{5}{0}{-1}{6/1/green}
		\ColorCellsdxdy{2}{1}{0}{6/8/purple}
		\ColorCellsdxdy{2}{0}{-1}{8/6/purple}
		\draw (0, 0) grid (\GridSize, \GridSize);
	\end{scope}
\end{tikzpicture}
\end{center}
Белые клетки --- нули, красные --- диагональ, одинакоые цвета показывают соответсвующие профили матрицы.\par
Данное представление хорошо подходит для $LU$ разложения, поскольку в его алгоритме вычисляются скалярные произведения строк на столбцы с равными индексами. Такой формат минимизирует число перемножений нулей и является оптимальным по кэшу для описанной операции.

\subsection{LU разложение}

\def\opn#1{\operatorname{#1}}
\newcommand{\mtop}[1]{%
\begin{tikzpicture}[#1]%
\draw (0,1.5ex) -- (1.5ex,0);%
\draw (1.5ex,0) -- (1.5ex,1.5ex);%
\draw (0,1.5ex) -- (1.5ex,1.5ex);%
\end{tikzpicture}%
}

\newcommand{\mbottom}[1]{%
\begin{tikzpicture}[#1]%
\draw (0,0) -- (1.5ex,0);%
\draw (1.5ex,0) -- (0,1.5ex);%
\draw (0,0) -- (0,1.5ex);%
\end{tikzpicture}%
}

$A=LU$\\
$U = \mtop{}$, \textbf{не} включает дагональ\\
$L = \mbottom{}$, включает диагональ\\
$L_{1, 1} = A_{1, 1}$\\
$\forall i:\; U_{i, i} = 1$\\
$\forall i\in\overline{2\dots n}: \forall j\in\overline{1\dots i-1}:$
$\begin{cases}
L_{i,j} = A_{i, j} - \left\langle\opn{row}_L(i), \opn{col}_U(j)\right\rangle\\
U_{j,i} = \frac{A_{i, j} - \left\langle\opn{row}_L(j), \opn{col}_U(i)\right\rangle}{L_{j, j}}\\
\end{cases}$\\

Для оптимизации объема выделяемой памяти результат складывается в матрицу $A$. Это возможно, поскольку на каждом шаге мы используем только ``старые'' значения матриц $L$ и $U$ и текущее значение матрицы $A$. Профильный формат хранения матрицы позволяет ``обрезать'' нули по краям строк и столбцов.

\section{Исследование погрешности}
Исследуем зависимость погрешности от числа обусловленности ($k$) и размерности пространства ($n$). Для этого будем генерировать матрицы коэффициентов за счет изменени диагонального преобладания для систем вида $A_kx_k=f_k$ следующим образом:\\
$\forall i\neq j:\; A_{i, j}\in\left\{0, -1, -2, -3, -4\right\}$\\
$A_{i,i}=\begin{cases}
-\sum_{i\neq j}A_{i, j} & i > 1\\
-\sum_{i\neq j}A_{i, j} + 10^{-k} & i = 1\\
\end{cases}$\\
Для оценки погрешности будем генерировать системы с заранее известным ответом: $f_k=A\cdot (1,\dots,n)^T$
\section{Скачкообразная погрешность}
\section{Матрицы Гильберта}
$\forall i,j\in\overline{1\dots n}:\; A_{i,j}=\frac{1}{i+j-1}$

\section{Вывод}

\newpage
\appendix
\section{Код}
Код матриц, $LU$ разложения и вся математика находится в директории \texttt{include/math} под говорящими названиями.
\subsection{Ссылки}
\url{https://github.com/kp2pml30/optimization-methods-labs}
\end{document}

