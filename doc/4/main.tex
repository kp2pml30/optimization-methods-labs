\documentclass[russian, english]{article}

% \usepackage[margin=1in]{geometry}
\usepackage[T2A]{fontenc}
\usepackage[utf8]{inputenc}
\usepackage[russian]{babel}
\usepackage{amsmath}
\usepackage{pgfplots}
\pgfplotsset{compat=1.9}
\usepackage{graphicx}
\usepackage{csvsimple}
\usepackage{hyperref}
\usepackage{verbatim}
\usepackage[pdf]{graphviz}
\usepackage[final]{pdfpages}

\inputencoding{utf8}
\def\code#1{\texttt{#1}}

\newcommand{\mydot}[2]{\left\langle#1,#2\right\rangle}

\newtoggle{aftersection}
\preto{\section}{\filbreak\global\toggletrue{aftersection}}
\preto{\subsection}{\iftoggle{aftersection}{\global\togglefalse{aftersection}}{\filbreak}}
\newcommand{\clearpageafterfirst}{%
  \gdef\clearpageafterfirst{\clearpage}%
}

\begin{document}

\begin{titlepage}
\centering
	{\scshape\LARGE Методы оптимизации \par}
	\vspace{1cm}
	{\scshape\Large Лабораторная работа №4\par Изучение алгоритмов метода Ньютона и его модификаций\par}
	\vspace{2cm}
	{\Large\itshape Дмитрий Проценко M3234 \par
	Кирилл Прокопенко M3236 \par
	Николай Холявин M3238 \par}
	\vfill
	ИТМО y2019
	\vfill
	{\large \today\par}
\end{titlepage}

\tableofcontents
\newpage

\section{Цель}
Исследовать методы Ньютона:
\begin{itemize}
	\item обычный
	\item с исчерпывающим спуском
	\item с направлением
\end{itemize}
Проанализировать:
\begin{itemize}
	\item траектории методов
	\item количество итераций
\end{itemize}

\section{Теория}
\subsection{Обозначения}
$x\otimes y\equiv x\cdot y^T$\\
$x^\otimes \equiv x\otimes x$\\
\subsection{Метод Ньютона}
Находит минимум унимодальной функции
\begin{itemize}
	\item[цикл] $x_{k+1} = x_{k} - \alpha p_{k}$
	\item[$\alpha$] пока $f(x - \alpha p) > f(x)$ делаем $\alpha\leftarrow \frac{\alpha}{2}$ (как в градиентном спуске)
	\item[$p$] $= H_f^{-1}(x)\cdot\operatorname{grad}_f(x)$
	\item[остановка] когда $x_k - x_{k - 1} < \varepsilon$
\end{itemize}
Для ускорения работы лучше не искать обратную матрицу, а решать СЛАУ $H_f(x)\cdot p=\operatorname{grad}_f(x)$
\subsection{С исчерпывающим спуском}
Вместо наивной аппроксимации $\alpha$ делает минимизацию одномерной функции от $\alpha$
\subsection{С направлением}
Не использует гессиан, вместо этого выбирается начальное направление $p_0$, на каждой итерации делается проверка:\\
$$
p = \begin{cases}
	p_k, & \mydot{p_k}{\operatorname{grad}_f(x)} > 0\\
	\operatorname{grad}_f(x), & \text{иначе}
\end{cases}
$$\\
Где $p_k$ определяется как раньше. \\
Идея метода в том, что он проверяет не идем ли мы в абсурдном направлении, и если это так, то он на этой итерации превращаетя в (градиентный) метод наискорейшего спуска.
\subsection{Квазиньютоновские методы}
Отличаются от классических методов тем, что не используют явную матрицу Гессе, а строят ее некоторое приближение.\\
$\omega_k = \operatorname{grad}_f(x_k)$\\
$p_k=G_k\cdot\omega_k$, где $G_k$ --- положительно определенная матрица ($n\times n$) специального вида.
\subsection{Метод Бройдена-Флетчера-Шено}
$G_1 = I$\\
$G_{k+1}=G_k - \frac{(\Delta x_k)^\otimes}{\mydot{\Delta \omega_k}{\Delta x_k}} - \frac{G_k(\Delta\omega_k)^\otimes G_k^T}{\rho_k}+\rho_k r_k^\otimes$\\
$r_k=\frac{G_k\Delta\omega_k}{\rho_k}-\frac{\Delta x_k}{\mydot{\Delta x_k}{\Delta\omega_k}}$\\
$\rho_k=\mydot{G_k\Delta\omega_k}{\Delta\omega_\Delta\omega_kk}$
\subsection{Метод Пауэлла}
$G_{k+1}=G_k-\frac{(\Delta \widetilde{x_k})^\otimes}{\mydot{\Delta\omega_k}{\Delta\widetilde{x_k}}}$\\
$\Delta \widetilde{x_k} = \Delta x_k + G_k\Delta\omega_k$

\section{Исследование методов Ньютона}

% function fancy-function prefix start [suffix/fancy]
\newcommand{\MakePlots}[5]{
	$z(x, y) = #2$\\
	$p_0 = #4$\\
	\begin{tikzpicture}
	\begin{axis}[
		colormap/greenyellow,
		legend style={at={(0.5,-0.1)},anchor=north},
		xlabel = {x},
		ylabel = {y},
		zlabel = {z},
		%colorbar
	]
		\addplot3[surf, opacity=0.8, forget plot] {#1};
		\foreach \suff/\fancy in #5 {
			\edef\temp{\noexpand\addlegendentry{\fancy}}
			\addplot3 table[col sep=tab]{#3/\suff.tsv};
			\temp
		}
	\end{axis}
	\end{tikzpicture}
}

\newcommand{\methodsList}{{}}
\begin{itemize}
	\renewcommand{\methodsList}{newton/{метод Ньютона},newton-with-minimization/{метод Ньютона с одномерной минимизацией}}
	\item \MakePlots{x^2 + y^2 - 1.2*x*y}{x^2 + y^2 - \frac{6}{5}xy}{test}{(4;1)}{\methodsList}
	\item \MakePlots{100(y-x^2)^2 + (1-x)^2}{100(y-x^2)^2 + (1-x)^2}{test}{(-1.2;1)}{\methodsList}
\end{itemize}

\section{Исследование методов квазиньютоновских методов}
\begin{itemize}
	\item \MakePlots{100(y-x^2)^2 + (1-x)^2}{100(y-x^2)^2 + (1-x)^2}{}{(-1.2;1)}{{}}
	\item \MakePlots{(x^2+y-11)^2 + (x+y^2-7)^2}{(x^2+y-11)^2 + (x+y^2-7)^2}{}{(-1.2;1)}{{}}
\item \MakePlots{100-2/(1+((x-1)/2)^2+((y-1)/3)^2)-1/(1+((x-2)/2)^2+((y-1)/3)^2)}
	{100-\frac{2}{1+(\frac{x-1}{2})^2+(\frac{y-1}{3})^2}-\frac{1}{1+(\frac{x-2}{2})^2+(\frac{y-1}{3})^2}}{}{(-1.2;1)}{{}}
\end{itemize}

\newpage
\appendix
\section{Код}
Основан на предыдущих лабораторных.

\section{Диаграмма классов}
Из-за использования концептов и шаблонов (в том числе CRTP), создающих нетривиальную иерархию классов, doxygen не сгенерирует адекватную диаграмму. Граф, сгенерированный вручную, представлен ниже. \\
\noindent\makebox[\textwidth]{\includegraphics[width=\paperwidth]{project.pdf}}
%\setboolean{@twoside}{false}
%\includepdf[pages=-,pagecommand={},width=\paperwidth]{project.pdf}

\end{document}
