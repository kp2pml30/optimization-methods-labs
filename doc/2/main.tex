\documentclass[russian, english]{article}

\usepackage[T2A]{fontenc}
\usepackage[utf8]{inputenc}
\usepackage[english,russian]{babel}
\usepackage{amsmath}
\usepackage{pgfplots}
\usepackage{graphicx}
\usepackage{csvsimple}
\usepackage{hyperref}
\usepackage[pdf]{graphviz}
\usepackage[final]{pdfpages}

\inputencoding{utf8}
\def\code#1{\texttt{#1}}

\begin{document}

\begin{titlepage}
\centering
	{\scshape\LARGE Методы оптимизации \par}
	\vspace{1cm}
	{\scshape\Large Лабораторная работа №2\par}
	\vspace{2cm}
	{\Large\itshape Дмитрий Проценко M3234 \par
	Кирилл Прокопенко M3236 \par
	Николай Холявин M3238 \par}
	\vfill
	ИТМО y2019
	\vfill
	{\large \today\par}
\end{titlepage}

\tableofcontents
\newpage

\section{Цель}
Исследовать методы:
\begin{itemize}
	\item градиентного спуска
	\item наискорейшего спуска
	\item сопряженных градиентов
\end{itemize}

\section{Примеры квадратичных функций}
\section{Зависимость числа итераций}
\subsection{Параметры}
\csvautotabular[separator=tab]{properties.tsv}
\subsection{Графики}
\pgfplotstableread{gradient-descent.tsv}{\ratiosCsv}
\begin{tikzpicture}[trim axis left]
	\begin{axis}[
		scale only axis,
		width=\textwidth,
		legend style={at={(0.5,-0.2)},anchor=north},
		xlabel = {число обусловленности},
		ylabel = {итерация},
	]
	\addplot table [x={cond}, y={n=10}] {\ratiosCsv};
	\end{axis}
\end{tikzpicture}

\section{Вывод}


\appendix
\section{Код}
Основано на первой лабораторной, исправлены некоторые архитекнутрные недостатки.

\section{Диаграмма классов}
Из-за использования концептов doxygen не сгенерирует адекватную диаграмму.
%\setboolean{@twoside}{false}
\includepdf[pages=-,pagecommand={},width=\paperwidth]{project.pdf}

\end{document}
